%%%%%%%%%%%%%%%%%%%%%%preface.tex%%%%%%%%%%%%%%%%%%%%%%%%%%%%%%%%%%%%%%%%%
% sample preface
%
% Use this file as a template for your own input.
%
%%%%%%%%%%%%%%%%%%%%%%%% Springer %%%%%%%%%%%%%%%%%%%%%%%%%%

\preface

NIME. A NIME. Two NIMEs. To NIME. Three NIMErs. Newcomers to the field often wonder about what the acronym NIME stands for. While the annual NIME conference is called \emph{International Conference on New Interfaces for Musical Expression}, we like that the acronym may take on several different meanings, such as: 

\begin{itemize}
	\item N = New, Novel, \ldots 
	\item I = Interfaces, Instruments, \ldots 
	\item M = Musical, Multimedial, \ldots 
	\item E = Expression, Exploration, \ldots
\end{itemize}

A little more than 15 years have passed since the small NIME workshop was held during the ACM Conference on Human Factors in Computing Systems (CHI) in 2001 \cite{Poupyrev:2001a}. Already from 2002, NIME was a conference on its own, and today it is an important annual meeting point of researchers, developers, designers, and artists from all over the world. The participants have very different backgrounds, but they all share a mutual passion for groundbreaking music and technology. 

More than 1200 papers have been published through the conference so far, and, staying true to the open and inclusive atmosphere of the community, all of the papers are freely available online.\footnote{http://www.nime.org/archive/} The archive is great if you know what to look for, but it has grown to a size that is difficult to handle for newcomers to the field. Even for long-timers and occasional visitors, it is difficult to get an overview of the history and development of the community. 

At recent editions of the conference we have seen a growing number of papers focusing on historical, theoretical and reflective studies of the NIME community (or even communities) itself. As this level of meta-studies started to grow, we began to see the potential for a collection of articles that could broadly represent the conference series. This thought has now materialised in the anthology you are currently holding in your hand, or reading on a screen. 


\section*{Who is the book for?}

The Reader consists of 30 selected papers from the conference series, reflecting the depth and width of the publication archive. Each paper is followed by two new commentaries that add value to the original works while at the same time bring to life some important underlying discussions. 

The anthology should be useful for several different groups of readers. Perhaps most importantly, we offer newcomers to the field an overview of some seminal works. They will find a broad range of topics and a large bibliography to continue their explorations outside the collection. We also hope that the book may be a useful reference point for researchers and artists who only ``visit'' the field and want a quick introduction and reference point to what is being done. Finally, we think that the book is relevant for longtime NIME participants. Such readers may be mostly interested in following historical traces and emerging critical reflections. 


\section*{The Selection Process}

By nature, an anthology is a limited selection of chapters, and for this project we settled on a limit of 30 articles from the NIME archive. This number was arrived at by estimating the page count of the resulting volume, and it also fitted well with the idea of including approximately two published items for each year of the NIME conference up to and including the most recent edition in 2015. 

As expected, selecting 30 papers from 1200 items was not an easy task. How could we possibly fairly represent the energetic and creative output in the field? From the outset, the NIME community has intentionally prioritized a diversity of research styles and approaches. The conference has also striven to offer an environment that can attract the participation not only of researchers working in an academic or institutional context, but also independent artists, researchers and inventors. Moreover, each of us has our own subjective interests and tastes as to what we consider significant prior work.

We started the selection process by creating an initial list of 50 or so articles perceived as ``influential'' by the community. This was accomplished by identifying the most-cited NIME papers using the Google Scholar index. Naturally, older papers tend to have more citations than new ones, so we also considered the number of citations-per-year. Both measures were used to create a ``multi-subjective'' initial list of works that have enjoyed impact.

From this first list of well-cited works we each separately created our lists of papers to include. In many cases we agreed fully, but for others, discussion and sometimes multiple discussions were needed before we reached agreement. At the same time, we found that the initial lists had gaps in the coverage of some topics. For example, work that is primarily artistic in nature is usually not as highly cited as technological reports. This led each of us to propose some works that have not been cited as highly as the others, but which we believe represent some of the diversity of the NIME corpus. Much discussion was also needed to reach agreement on which of these was to be included in the final list.

Needless to say, we have not been able to include all of the papers that we believe deserve a place in a NIME anthology. The final selection reflects a compromise to respect the limit of just 30 articles. This is also why it is consciously titled ``\emph{A} NIME Reader'' and not ``\emph{The} NIME Reader.'' We hope that there will be other anthology projects drawing on the extensive NIME proceedings in the future. 


\section*{Peer-Commentary Process}

It was clear from the outset that further peer-review of the articles was not necessary, all of the articles had already undergone peer-review in order to be published at the NIME conference. However, an important feature of this Reader is the inclusion of commentary by the authors themselves as well as other knowledgeable participants in the NIME community (henceforth ``experts'' or ``peers''). Commentators were selected from the group of active conference participants who were perceived to have a special interest in given articles. Authors and peers had a chance to exchange feedback on each other's commentaries. The feedback ranged from simple corrections to enthusiastic discussions of philosophical issues. The peer-commentary procedure, inspired by Stevan Harnad \cite{Harnad:2000}, really brought this project to life. The exchange between authors and experts illustrated how the articles continue to live, function, and contribute growth in the research community.

\section*{How can NIME Continue to be Relevant?}

Our hope is that this anthology project will encourage reflection within the community and nurture ideas that have been presented at the conferences. Many NIME papers are fairly terse, with room to present only one, or a few, core ideas of a larger project. This book may provide a starting point for expanding, connecting, and developing new research directions. Originally, NIME was a spin-off from the CHI conference, and recently other events are spawning from NIME. This shows that the field is alive and blossoming, a healthy sign that it will continue to be relevant  for another 15 years. 

It is important to stress that this collection is not intended to be \emph{the} ultimate NIME paper selection, or a definitive ``canon'' of relevant works. Going through the archives, we were astounded by the breadth, width and quality of ideas presented, many of which deserve more attention. We would encourage others to conduct meta-studies, write review papers and create further anthologies. The archive is large enough to consider more specialized anthologies focusing on, for example, mobile or web technologies, machine learning, tactility, performance, and pedagogy, to name only a few. 

In conclusion, we hope that this collection will be the first of many. After all, the NIME archive is a gold mine of good ideas, and a history of experience and knowledge gained through the dedicated work of many talented researchers and artists. These should not be forgotten but continue to be used in different ways. They may even spark entirely new ways of thinking about NIME itself.


\vspace{\baselineskip}
\begin{flushright}\noindent
Brisbane,\hfill {\it Alexander Refsum Jensenius}\\
July 2016 \hfill {\it Michael J. Lyons}\\
\end{flushright}


