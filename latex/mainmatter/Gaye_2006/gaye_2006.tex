
\graphicspath{ {mainmatter/Gaye_2006/} }
\title*{2006: Mobile Music Technology: Report on an Emerging Community}
\titlerunning{Mobile Music Technology}

\author{Lalya Gaye, Lars Erik Holmquist, Frauke Behrendt and Atau Tanaka}
\authorrunning{Gaye et al.}

%\institute{Lalya Gaye \at Attaya Projects, Newcastle upon Tyne, \email{lalya@attayaprojects.com}
%\and Lars Erik Holmquist \at Grounded Innovation, Tokyo, JAPAN  
%\and Frauke Behrendt \at Media Studies, University of Brighton, UK \email{f.behrendt@brighton.ac.uk}
%\and Atau Tanaka \at Embodied AudioVisual Interaction Group, Goldsmiths, London, UK \email{atau@goldsmithsdigital.com}}
%
%
\maketitle

\abstract*{The new field of mobile music emerges at the intersection of ubiquitous
computing, portable audio technology and NIME. We have held a series of
international workshop on this topic with leading projects and speakers, in order
to establish a community and stimulate the development of the field. In this
report, we define mobile music, and map out the field by reporting on the
workshop series and accounting for the state-of-the-art.}

\section{Introduction}

\textit{Mobile music} is a new field concerned with musical interaction in
mobile settings, using portable technology. It goes beyond today's portable music
players to include mobile music making, sharing and mixing. The core themes of
NIME---interaction, interfaces and music, can today be deployed on mobile
electronics. While NIME projects have mostly been concerned with stationary
concert performance or installations, mobility allows NIME concepts to occupy
exterior urban space, and exploit people's movements through it, as well as the
heterogeneous space and social dynamics found in those environments. The
\textit{International Workshops on Mobile Music Technology} are the first events
that focus on this new field. They have played a key role in the development of
mobile music since 2004 and can be regarded as one direction for expansion of the
NIME community. This report establishes a definition of mobile music, describes
the workshop series, and accounts for the state-of-the-art.

\section{Mobile Music Technology}

\begin{figure}[t]
\centering
\includegraphics[width=\textwidth]{img1.pdf}
\caption{Mobile music projects: (a) Sonic City and (b) Malleable Mobile Music.}
\label{Gaye:fig:1} 
\end{figure}

A number of recent technological
advances have pushed the envelope of possible human-computer interactions, giving
rise to new fields such as locative media and pervasive gaming. Mobile computing
enables systems to be used anywhere and on the move. Coupled with context-aware
computing and global positioning technology, mobile devices can respond to the
user's surroundings and location, situating interaction within everyday settings.
Augmented and mixed-reality technologies merge the digital and physical realm,
making them cohabit in the same environment. Ad hoc, peer-to-peer and distributed
networking allow groups of users---ranging from co-located strangers to
dislocated friends---to collaborate spontaneously, across distances and without
the need for centralised supervision. Meanwhile, the miniaturisation of consumer
electronics and improvements in high-capacity digital storage have given rise to
powerful portable mp3 players that could easily contain one's complete music
collection. Moreover, mobile phones have brought us ubiquitous network
connectivity. Mobile music emerges at the crossroads of these technological
advances and their resulting new practices, joining the worlds of ubiquitous
computing and locative arts, with mobile consumer electronics, and the sensing
and interaction tradition of NIME.

\subsection{Mobile Music: Beyond Portability}

Mobile music as a term covers any musical activity using portable devices that
are not tethered to a specific stationary locale; in particular those where the
activity dynamically follows users and takes advantage of the mobile setting,
thereby leveraging novel forms of musical experience. Mobile music devices might
possess properties such as context awareness, ad hoc or distributed network
connectivity, or location sensing, sometimes combined with technology embedded in
the physical environment. Therefore, they can be used anywhere and on the move,
and take advantage of people's displacements, location, and of the changes of
social and geographical context that mobility implies. Examples of mobile music
activities include pushing music to people nearby \cite{Jacobsson:2005}, sonifying local Wi-Fi
coverage while riding a bike \cite{McCallum:2005}, or remixing music tracks with remote friends
across peer-to-peer networks \cite{Tanaka:2005}. Mobile music goes beyond the iconic
Walkman\texttrademark{}, and does not need to imply individual use, headphones or
passive music listening. It spreads over a large spectrum of musical
interactions, ranging from consumption to creation, and with mobility
increasingly blurring this distinction \cite{DArcangelo:2005}. Mobile music resonates with
practices of both NIME musicians and everyday users of consumer audio products.

\subsection{Reconsidering Musical Interaction with Mobility}

Mobile music creates a tension between music and place as well as new
relationships between musician, listener, and music. For electronic musicians,
the mobile environment offers more than just a new place to transplant NIME
techniques. Rather, mobility encompasses specificities that encourage us to
reconsider the basic tenets of musical interaction. Mobility implies outdoor
environments where space and place become tangible parameters, and also implies
always-on itinerant devices: location can become a ``sensor'' input to music
systems, people nearby can become part of an ad hoc networked musical
performance. With networked multi-user systems, mobility allows musical
engagement beyond eye-to-eye contact. It also asks the NIME musician to consider
social aspects in everyday public space, an environment not primarily dedicated
to music use and where people might already be involved in a number of adjacent
and simultaneous activities.

\subsection{Another Dimension to Creative Engagement with Consumer Products}

Many mobile music projects draw on earlier popular electronic music movements
such as remix- and DJ-culture, file-sharing or playlists. They extend creative
ways of engaging with portable consumer audio technology by weaving them into
ever-changing geographical and social contexts. One example is tunA \cite{Bassoli:2004}, where
people in close proximity can share music by listening to each other's mp3
playlists, getting a taste of people's musical preferences across various social
situations. There is a broad range of possibilities in terms of making music with
mobile consumer devices, from ringtones, to mobile soundscape recording or sound
art. Widespread platforms such as mobile phones are used as musical instruments
and interfaces, encouraging the public to explore new ways of looking at their
personal mobile devices. Projects working with such communication technologies
invite musicians and lay people alike to participate in performances, group
improvisation, sound art or remixing, for example collaborating with strangers in
same physical space (e.g. on the bus), or jamming with remote friends while
strolling around town.

\section{International Workshops}

At its early stage, mobile music was rapidly gaining popularity and relevance
but lacked a clear sense community and an explicit demarcation as a field. As for
any emerging field, it was therefore important to establish a community of people
who could share experiences and communicate ideas. A good way to achieve such a
goal is focused workshops. For instance, the NIME conference series grew out of a
workshop at CHI 2001 \cite{Poupyrev:2001a}. In 2004, we started a series of international
workshops on the subject in order to establish and develop the field of mobile
music. The workshops have gathered a mix of researchers, designers, musicians,
new media artists, and representatives of the industry. They have raised
awareness about existing projects as well as helped actors of the field with
backgrounds in multiple disciplines to identify common goals and issues, share
resources, and introduce one another to relevant technologies, methods and
concepts. The purposes and formats of the workshops have varied as the community
evolves but activities in common include presentations of projects, in-depth
discussions, brainstorming sessions and hands-on activities.

The first \textit{International Workshop on Mobile Music Technology }was
organised at the Viktoria Institute in G\"{o}teborg, Sweden, in June 2004. The
purpose was to gather a number of researchers with a shared interest in mobile
music, and to attract additional people who might be interested in making the
community grow. This workshop focused on presenting existing projects and
defining the field. It had 15 external participants, plus organisers and student
volunteers.

The second workshop was organised in May 2005 at NIME 2005 in Vancouver, Canada.
This time, the community was better defined, and the workshop time was shared
between presentations of new projects, in depth-discussions and hands-on
brainstorming activities. It attracted 18 external participants.

The third edition of the workshop was a two-day event that took place in March
2006 at the University of Sussex, Brighton, UK. It gathered nearly 30
participants and focused on the locative media aspect of mobile music, with
presentations by invited speakers, feedback sessions about work-in-progress
projects, and hands-on activities with the latest mobile music technology.

\subsection{Projects}

The workshops feature state-of-the-art mobile music projects, in the form of
presentations by guest speakers and peer-reviewed papers, posters and
demonstration sessions (see Figure~\ref{Gaye:fig:1}.a), as well as feedback sessions for
works-in-progress. These projects are at the centre of the field's development
and demonstrate its diversity and potential. Many use generic mobile platforms
such as mobile phones or handheld computers; others use hacked or custom-made
technology to better respond to specific needs and requirements. All have in
common taking advantage of the mobile nature of mobile technologies and
situations as an intrinsic part of their work. Projects can be grouped along the
following emerging themes.

Several projects explore \textit{collaborative music making} with mobile
technology. Malleable Mobile Music \cite{Tanaka:2004} (see Figure~\ref{Gaye:fig:1}.b) is a location-based and
peer-to-peer networked remixing system. TGarden \cite{Ryan:2003} is an interactive
environment for theatrical music making using wearables. Sequencer404 \cite{Jimison:2006} allows
multi-user control of a musical sequencer through telephony and Voice over
Internet Protocol (VoIP). In CELLPHONIA \cite{Bull:2006}, people engage in a location-based
mobile phone karaoke opera. The collaborative public art performance China Gates
 \cite{Clay:2006} synchronises a set of tuned gongs with GPS as participants follow different
routes. Mobile phones are used for interacting with a sound installation in
Intelligent Streets \cite{Lorstad:2004}. Finally, IMPROVe \cite{Widerberg:2006} is a architecture for
collaborative improvisation with sounds recorded with mobile devices.

Some of the projects in the genre of \textit{mobile music making} enable
individual users to manipulate sounds and create music by \textit{interacting
with environmental factors}: the physical urban environment with Sonic City \cite{Gaye:2003}
(see Figure~\ref{Gaye:fig:1}.a), and ambient lighting conditions in Solarcoustics \cite{Barnard:2005}. A mobile
user-interface platform for such interactions in a personal area network (PAN)
was also demonstrated \cite{Yamauchi:2005}.

Another theme is \textit{mobile music listening and sharing}. Some projects
address the sharing of playlists and music across peer-to-peer networks, enabling
users to listen to their neighbours' music either synchronously (SoundPryer \cite{Ostergren:2004}
and tunA \cite{Bassoli:2004}) or asynchronously (Push!Music \cite{Jacobsson:2005}). Other projects transform music
albums into narratives spread across geographical space (Location33 \cite{Carter:2005}), or
enable the cultivation of public ``sound gardens'' located in Wi-Fi connections,
as an overlay on physical space (Tactical Sound Garden [TSG] \cite{Shepard:2006}).

A third area is dedicated to \textit{HCI and mobile music}. It includes
SonicPulse \cite{Anttila:2006}---a project providing an acoustic way of passively monitoring or
actively exploring a shared music space, Music Mood Wheel \cite{Andric:2006}---an auditive
interface for navigating music spaces on the move, and Minimal Attention
Navigation via Adapted Music \cite{Hunt:2006}---a musical navigation system for pedestrians.

Meanwhile, some workshop participants have taken a more sociological or
media-studies approach, looking at the relation between music taste, use and
identity \cite{DArcangelo:2005}, soundscapes and people's everyday experience of place \cite{Phillips:2006}, mobile
phones and its use in sound-art \cite{Behrendt:2004}, and mobility, sound and urban culture \cite{Bull:2001}.
These contributions have brought insightful humanistic perspectives for the
development of the field of mobile music, grounding it on social realities,
aesthetics and already emerging practices.

\subsection{Group Activities}

The workshops included group activities with structured brainstorming sessions
in the first two workshops, as well as feedback sessions and hands-on experience
of mobile music technology in the third one. These were combined with in-depth
discussions on various topics relevant to mobile music and on current issues,
opportunities and challenges in the field---e.g. the relationship between mobile
devices, space and the body in movement, or how to approach context-aware
platforms developed in the field of ubiquitous computing with a NIME perspective.

\subsubsection{Brainstorming Sessions in the 1$^{st}$ Workshop}

An important function of the first workshop was to map out the field and define
future directions. We organised a series of structured brainstorming activities
that ran over two days. Participants were divided into three groups, each
dedicated to one of the following topics: \textit{mobile music creation};
\textit{mobile music sharing}; \textit{business models and the future of the
mobile music industry}. In addition to this, we had pre-defined a number of
themes to investigate: \textit{infrastructure and distribution}; \textit{genre
and formats}; \textit{social implications}; \textit{ownership}; \textit{business
models}; \textit{creativity}; \textit{interaction and expression};
\textit{mobility}; \textit{users and uses}.

Each group would choose four themes from the list, and discussed them from the
perspective of their overall topic. For instance, the group on \textit{Business
models }discussed the theme \textit{Genre and format, }raising issues such as
length of compositions, use of meta-tags or potential revenue from different
kinds of formats. After the first day of brainstorming, results were presented to
the other groups. The second day was dedicated to defining design dimensions for
mobile music applications based on day one's sessions (for example solo vs.
collective, foreground vs. background), and to mapping the emerging design space
to existing or future projects. The sessions raised a number of issues, including
``in-between'' states that are neither mobile nor stationary, how musical taste
is used to establish personal identity, to the meaning of ownership and where
added value could be elicited.

\subsubsection{Bodystorming Session in 2$^{nd}$ Workshop}

\begin{figure}[t]
\centering
\includegraphics[width=\textwidth]{img2.pdf}
\caption{Mobile music workshops: (a) Project demonstration and (b) scenario Body-storming.}
\label{Gaye:fig:2} 
\end{figure}

In the 2$^{nd}$ workshop, hands-on
activities were kept to one afternoon. They focused on bodystorming of mobile
music applications and scenarios. Bodystorming is a method where participants act
out a particular scenario of use, taking the roles of e.g. users or artefacts and
focusing on the interaction between them \cite{Buchenau:2000}. With this method, participants
explored various mobile music themes, developed simple application ideas, and
physically enacted scenarios of use in order to get an embodied understanding of
design challenges and opportunities specific to mobile music.

Participants first combined randomly chosen instances of the following
categories: \textit{situations} (e.g. driving a car while it snows); users (e.g.
school kids); \textit{technological infrastructures} (e.g. Wi-Fi, GPS);
\textit{types of music uses} (create, share, organise...). Combinations were
assigned to each group and developed into 3 application or scenario ideas per
group during short brainstorming sessions. Each group decided on one idea and
further developed it through bodystorming. Scenarios were then acted out to the
rest of the workshop to stimulate discussion. An example of scenario was a
bicycle-taxi working as a peer-to-peer server and broadcasting its clients' music
on loudspeakers in Kingston, Jamaica (see Figure~\ref{Gaye:fig:2}.b). This scenario generated
discussions on mobile ways of sharing and outputting music in public space, and
of their social adequacy.

\subsubsection{Feedback Sessions and Hands-On Activities in the 3$^{rd}$
Workshop}

On the first day of the third workshop, selected work-in-progress projects grouped in parallel sessions received expert feedback during critical and supporting discussions. Through this participants identified crucial issues and presented their findings to the other groups. The second day was hands-on and gave participants access to technologies for mobile music that they might otherwise not have been exposed to. Participants were given tutorials on sensors for mobile music, and on miniMIXA, a mobile music software mixer and mini recording studio for hand-held devices. They were also introduced to socialight, an audio space annotation platform for sharing location-based media. As a follow-up, participants sketched out possible applications combining such technologies.

\subsection{Dissemination of Results}

The output from the workshops has been presented in contexts outside of NIME and
of the workshop itself. Two of the co-authors moderated a panel discussion at the
ACM SIGGRAPH 2005 conference on the subject of \textit{Ubiquitous Music} \cite{Holmquist:2005}
where the majority of the panellists selected by the conference had previously
participated in the workshops. As the largest international conference on digital
media and emerging technologies, the SIGGRAPH panel underscored the pertinence of
mobility and musical interaction to a wider field. Authors have also given
lectures about mobile music in art and design schools. Currently, the results
from the first workshop are being edited into a book that will be a key reference
emphasising the creative potential of mobile music technologies.

\section{Conclusions and Future Work}

We have presented the field of mobile music, its current state-of-the-art, as
well as a workshop series with a decisive influence on its development. During
the workshops, a multitude of emerging key topics concerning the socio-cultural,
artistic, technological and economical aspects of mobile music have been
identified. The overall experience of these events has been very positive. Out of
the participants has crystallised a core group, which is very active in the
field. The community continues to grow, with new people being attracted to each
workshop and the number of relevant projects increasing consistently. The future
of mobile music is now being shaped by a collective community effort and promises
interesting future developments. In order to further extend and consolidate the
mobile music community and support these developments, we will continue to
organise new workshops and will soon publish a website as a resource for mobile
music projects and related publications.

\begin{acknowledgement}
The 1$^{st}$ workshop was organised by the Viktoria Institute, the 2$^{nd}$ by
Viktoria and Sony CSL Paris (in conjunction with NIME'05), and the 3$^{rd}$
workshop was a collaboration between Viktoria, the Universities of Sussex and
Salford, the Pervasive and Locative Arts Network (PLAN) and Futuresonic. We wish
to thank our workshop co-organiser Drew Hemment, the NIME'05 chairs Tina Blaine
and Sidney Fels, as well as the mobile music core group, including among others
Arianna Bassoli, Gideon d'Arcangelo, Maria H\aa{}kansson, Rob Rampley, Chris
Salter and Mattias \"{O}stergren. We also warmly thank all the workshop
participants, reviewers and student volunteers, for making this series of events
successful.
\end{acknowledgement}

\section*{Author Commentary: Mobile Music Technology: from Innovation to Ubiquitous Use}
\paragraph{Frauke Behrendt}

A time before smartphones becomes more difficult to imagine by the day, a time before digital, networked, sensor-studded personal mobile devices became ubiquitous. In such a time, more specifically in 2004, drawing on music's rich history of mobility and responding to emerging developments and innovations in mobile technology, I was part of an interdisciplinary group of researchers and designers that came together to experiment with and analyse mobility and music in the context of increasingly ubiquitous networked devices, and we thus contributed to establishing the field of mobile music.

We organised four \lq International Mobile Music (Technology) Workshops\rq  between 2004 and 2008, in Gothenburg, Vancouver, Brighton and Vienna. These are mentioned in the paper and documented in more detail in the book accompanying the final event \cite{Kirisits:2008}. By 2008, mobile music had become mainstream and an integral part of several fields of research and practice, including NIME and app design culture.

Our 2006 paper on the emerging community around mobile music technology has been used and developed in a range of research areas, by researchers and designers from around the world, both within and beyond the NIME community, as becomes evident from reviewing the papers and patent citing the paper. The main contribution of the paper has been for those designing mobile music products or services, such as mobile phone apps, software or hardware, for example \cite{Wang:2009}. Almost equal interest has come from those designing or evaluating interactive and/or collaborative performances with mobile phones for performers and for audience participation (e.g. mobile phone orchestras or social music making platforms). Overviews, classifications and taxonomies of the field of mobile music from various perspectives have also drawn on the work, for example by considering the social, cultural and historic dimensions of mobile music \cite{Gopinath:2014}. The field of sonic interaction design \cite{Rocchesso:2008}, the field of locative music and sound (e.g. GPS sound walks), the educational use of mobile music (e.g. mobile phone music learning for children), sound studies (e.g. ubiquitous listening) and media studies (e.g. global mobile media), are other areas where the paper has made a contribution.

My own research contributions on mobile music technology include developing a taxonomy of mobile music with four categories: musical instruments, sonified mobility, sound platforms and placed sound. These categories were explored in more detail through a number of detailed analysis of specific artworks and apps, drawing on empirical material gathered through interviews, observations, ethnographies and case studies. This research material was analysed in light of theories and concepts from media studies, mobility studies, NIME and sonic interaction design \cite{Behrendt:2015}. More recently, I have drawn on the field of mobile music as research partner on the NetPark project\footnote{\url{http://www.metalculture.com/projects/netpark/}} that turned a public park into an ongoing and growing collection of mobile and locative artworks, many of which focus on sound and music. This presents a platform for ongoing research on both the design process and the audience perspective/user experience of the NetPark and the works hosted and curated in it. There is also a close relation between this most recent engagement with mobile music technology and my other research around mobile media, smart cities, the Internet of Things and sustainable mobility, in that all my work considers mobility and musical/sonic perspective, in part inspired by this NIME paper. Over time, my engagement with the mobile music technology community has shifted from a more technical perspective and an active engagement in the NIME community towards a more theoretical and empirical analysis of the social and cultural aspects of mobile music in the field of media studies.
In the years since our early community and the series of workshops on mobile music technology, mobile music has become so ubiquitous that the topic is now well-established in a range of research and practice communities.

\section*{Expert Commentary: Mobile music making paradigm: towards a new culture of use}
\paragraph{Koray Tahiro\u{g}lu}

Although the primary focus in the first era of mobile music research was on the ways in which mobile devices raised unique opportunities in locative media, it is clear from a historical perspective that the actual goal was to establish mobile music making as a research field and to create scientific and artistic legitimacy around it. This was  achieved by bringing together the NIME community's early adopters  to explore mobile technologies as new platforms for music making. The idea of organising workshops was successful in getting the attention of the musicians,  designers,  researchers and industry people  who shared interests common with the fundamental NIME approach to music, interaction and technology. These workshops, where the projects, ideas and concepts of practitioners were introduced and shared, were the first steps taken towards establishing a platform for mobile music. It is important to remember that early workshops were organised before the first generation of smartphones. Regardless of the mobile technology specifications, the creative and interactive focus  differentiated mobile music from existing forms of practices and presented possibilities that were clearly distinct from traditional interactive music systems.

The early workshops presented ideas for some aspects of future developments in the field, such as possible social-music experiences and interaction models that question the roles of artists and listeners in the creative process. However, the future of music practise using mobile platforms was explored only with general statements and its relation to the relative economic, cultural and mediated paradigms was barely considered. Many of the projects presented for new ways of creating music focused significantly on the design constraints of working with state-of-art technology. The music industry was undergoing a transitional period at the time, as it can be argued it still is, so the workshops could have explored the technological context of mobile music making from a wider perspective; consciously evaluating the economic, cultural and social factors in the way musicians have always had to. Perhaps proposing tentative hypotheses on the evolution of mobile music for a period of time when mobile devices have developed beyond portable-playback devices to smart systems, could have offered more insight into future directions.

Nevertheless, after the first era of mobile music research, the collective community continued in its efforts to  reflect on and share the developments in the field of mobile music. For instance, the Designing Musical Interactions for Mobile Systems workshop was organised in order to discuss the specific interaction design challenges for deploying engaging and creative musical activities on mobile devices in the smartphone era \cite{Tahiroglu:2012}.  Simultaneously, the explosion of commercial music apps has directed the industry and the research to the widespread potential of music on mobile devices. During the workshop these different categories of use were discussed in detail with a set of  interface design models for  \textit{music instruments, controllers, portable studios, game and ambient interactions, social / network components}  for creating rich musical interactions that push the capabilities of present day mobile phone technologies \cite{Tahiroglu:2012}.

Mobile music making holds a special place in social-interactive aspects of research in NIME community \cite{Bryan-Kinns:2004,Yang:2015}. Mobile phone technology  supports  mobile and casual music-playing, facilitating interactive performances \cite{Wang:2009}.  Current smartphones are powerful, network connected and equipped with Audio I/O, touchscreens, cameras and other embedded sensor input mechanisms. The increasing processor power of mobile devices makes real time signal processing and sound synthesis possible, enabling advanced music composition and performance tasks to be carried out on a mobile device. 
Most importantly,  mobile devices advance opportunities for interaction in a collaborative context and have created a culture in music practices that was unlikely foreseen.
In order to envision new strategies for mobile music making that could allow mobile technology to expand the meaning of ``mobile,'' it is worth considering the ways the community has defined work methods, practices  and criteria for musical expressivity \cite{Tanaka:2012}.

It is important to be aware of the unrealised potential of mobile music making. More design work needs to be done in order to explore the full potential of mobile technologies, the different ways in which user interfaces can be manipulated and the gestural capabilities of the devices. 
Furthermore, it is critical to consider the paradigms of computer music, ``real world'' instruments and the listening experience within an appropriately broad view of musical interactions. This can be achieved by giving equal weight to the performer, audience, technologies and cultural forces when making them mobile.
