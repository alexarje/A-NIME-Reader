
\graphicspath{ {mainmatter/Jensenius_2014/} }

\title*{2014: To Gesture or Not? An Analysis of Terminology in NIME Proceedings 2001--2013}
\titlerunning{To Gesture or Not?} %for an abbreviated version of
% your contribution title if the original one is too long

\author{Alexander Refsum Jensenius}
\authorrunning{Jensenius} %for an abbreviated version of
% your contribution title if the original one is too long

%\institute{Alexander Refsum Jensenius \at fourMs group, Department of Musicology, University of Oslo, PB 1017 Blindern, 0363 Oslo, Norway \email{a.r.jensenius@imv.uio.no}}
%
%
\maketitle

\abstract*{The term `gesture' has represented a buzzword in the NIME community since the beginning of its conference series. But how often is it actually used, what is it used to describe, and how does its usage here differ from its usage in other fields of study? This paper presents a linguistic analysis of the motion-related terminology used in all of the papers published in the NIME conference proceedings to date (2001--2013). The results show that `gesture' is in fact used in 62~\% of all NIME papers, which is a significantly higher percentage than in other music conferences (ICMC and SMC), and much more frequently than it is used in the HCI and biomechanics communities. The results from a collocation analysis support the claim that `gesture' is used broadly in the NIME community, and indicate that it ranges from the description of concrete human motion and system control to quite  metaphorical applications.}

\section{Introduction}
\label{Jensenius:introduction}

How we talk about the things we do matters. All artistic communities and research fields have their own jargon, their own buzzwords, and their own way of expressing things. This helps to create a sense of common ground or purpose within the given community, and it can be important in terms of  differentiating oneself from others. But the terminology used \emph{within} a community also forms the basis for communication with people \emph{outside} of it. For such interdisciplinary dialogue, it is important to carefully define one's terminology, so that other artists and researchers can follow one's discussions. 

Ever since I started attending the NIME conferences back in 2005, I have been struck by the widespread use of the term `gesture' within this community. There is nothing wrong with the term in itself, but it is striking that its usage has not been discussed more. It also appears that `gesture' is often used without being properly defined, as though its meaning were obvious or straightforward. In fact, I have come to find that its explicit and implicit definitions are quite diverse, and range from its use as more or less synonymous with body motion to more purely metaphorical senses. 

The issue, then, is that we might well become confused within the community, but we might become even more so when we interact with people in other fields of study---for example, physiotherapists, researchers of biomechanics, linguists, or even musicians. Many of these scholars do not understand why we use `gesture' to describe phenomena for which they have other words. 

Interestingly, while I have long had a \emph{feeling} that `gesture' is used quite liberally at NIME, I have had no proof of it. This paper therefore presents a linguistic analysis, based on the papers published in the NIME proceedings, that aims to answer the following questions: 

\begin{enumerate}
\item How much is `gesture' used at NIME? 
\item How much is `gesture' used in related fields?
\item How is `gesture' used, and with what meaning(s)? 
\end{enumerate}

The paper starts with a review of some definitions of the term. Next is a presentation of the analytical approach taken, based on a linguistic corpus analysis, followed by a presentation and discussion of the findings. 



%%%%%%%%%%%%%%%%%%%%%%%%%%%%%%%%%%%%%%%%%%%%
\section{Gesture Definitions}
\label{Jensenius:definitions}

Before delving into the analysis, I will review both dictionary-type definitions of `gesture' and more specific definitions from the academic literature. 


%%%%%%%%%%%%%%%%%%%%%%%%%%%%%%%%%%%%%%%%%%%%
\subsection{Dictionary Definitions}

The Oxford dictionary %\footnote{http://www.oxforddictionaries.com/definition/english/gesture} 
offers a classic definition: 

\begin{quotation}
\emph{a movement of part of the body, especially a hand or the head, to express an idea or meaning}
\end{quotation}

\noindent
This definition is almost identical to those of other large dictionaries, including Merriam-Webster, %\footnote{http://www.merriam-webster.com/dictionary/gesture}  
Collins %\footnote{http://www.collinsdictionary.com/dictionary/english/gesture} 
and Dictionary. %\footnote{http://dictionary.reference.com/browse/gesture} 
It is interesting to note that all of these definitions focus on three elements: 

\begin{itemize}
	\item movement of the body
	\item in particular, movement of the hands or head
	\item expression of an idea/meaning/feeling
\end{itemize}

\noindent
The MacMillan dictionary %\footnote{http://www.macmillandictionary.com/dictionary/british/gesture} 
adopts a %slightly 
broader definition:

\begin{quotation}
\emph{a movement that communicates a feeling or instruction}
\end{quotation}

\noindent
Here, `instruction' has been added as part of the definition, and this is also followed up with two sub-definitions: 

\begin{quotation}

\emph{a. hand movement that you use to control something such as a smartphone or tablet [...]}

\emph{b. the use of movement to communicate, especially in dance}

\end{quotation}

\noindent
Of all of the general definitions of `gesture,' MacMillan's definitely resonates best with the NIME community's use of the term.


%%%%%%%%%%%%%%%%%%%%%%%%%%%%%%%%%%%%%%%%%%%%
\subsection{Academic Definitions}

There have been several review articles concerning the use of `gesture' in music, including  \cite{Cadoz:2000,Jensenius:2010}. The latter \cite{Jensenius:2010} groups the different definitions of `gesture' into three main categories:

\begin{itemize}
	\item Communication: gestures are used to convey meaning in social interaction (linguistics, psychology)
	\item Control: gestures are used to interact with a computer-based system (HCI, computer music)
	\item Metaphor: gestures are used to project movement and sound (and other phenomena) to cultural topics (cognitive science, psychology, musicology)
\end{itemize}

The first type of definition most closely resembles the general understanding of the term, as well as the definition that is presented in most dictionaries. The second type represents an extension of the first, but incorporates a shift of communicative focus from human--human to human--computer communication. Still, the main point is that of the conveyance of some kind of meaning (or information) through physical body motion. The third type, on the other hand, focuses on `gesture' in a metaphorical sense. This is what is commonly used when people talk about the `musical gesture.' The problem, however, is that the use of `musical gesture' drifts widely, as can be seen in some important publications from the last decade \cite{Godoy:2010,Gritten:2006,Gritten:2011,Hatten:2004}. 

While there are no problems with the definition types in themselves, and even with the sub-definitions within each of the three main groups, I see the potential for confusion when the term is not explicitly aligned to one of them when it is used. This is particularly so in the NIME community, because NIME gathers artists and researchers who are working at the intersection between HCI and music(ology), within which two very different types of gesture definitions are commonly evoked. 

From an HCI perspective, `gesture' has been embraced as a term to describe bodily interaction with computing systems. In its purest sense, such as finger control on a touchscreen, this type of human--computer communication is not especially different from that of the `gesture' used in human--human communication. Likewise, nowadays most people are accustomed to controlling their mobile devices through `pinching,' `swiping,' etc., so it seems like such `HCI gestures' have become part of everyday language, just as the MacMillan definition suggests. 

Staying within the HCI ecosphere, the picture becomes slightly more complex when one starts talking about `expressive gestures.' This can refer to the conveyance of some emotional state in multimodal interaction \cite{Camurri:2002}, or describe
large and complex vocabularies of short and simple bodily actions. Such definitions, however, may not be as contradictory to traditional gesture definitions as one might think. After all, expressing emotional quality is also an important element of traditional hand gesturing \cite{Lawson:1973,McNeill:1992}.


Moving on to the metaphorical type of definition, `musical gesture' has become a popular way to describe various types of motion-like qualities in the perceived sound \cite{Godoy:2010} or even in the musical score alone \cite{Hatten:2004}. This, obviously, is a long way from how `gesture' is used to evoke a meaning-bearing body motion in linguistics, although it may be argued that there are some motion-like qualities in what is being conveyed in the musical sound as well. I will not delve deeper into the epistemological challenges of the term `musical gesture' here, but I will point to a recent philosophical enquiry into this specific term \cite{Funk:2013}. The following sections will instead focus on `gesture' and body motion, as I see this relation as the main issue regarding how people outside our field confuse the way `gesture' is used within it. 


\section{`Gesture' in NIME Proceedings}

To investigate the usage of `gesture' in the NIME community, I decided to carry out a linguistic analysis based on all of the papers published at the NIME conferences. 


\subsection{Method}

The first step in the analysis was to download PDF files of all of the papers from the freely available NIME proceedings archive.\footnote{\url{http://www.nime.org/archive/}} After running a PDF consistency check in Adobe Acrobat Pro, three files were found not to contain searchable text. Alternate PDF files of two of these papers were found online and replaced in the corpus. The last defective PDF file was removed from the corpus, leaving a total of 1,108 files to be analysed (see Table~\ref{Jensenius:tab:NIME} for yearly distribution).

Next, I defined a set of search terms. I used `music' as a control term, because I expected it to show up in all of the papers. In addition to `gesture' itself, I included terms that somehow overlap with, or are used together with, `gesture': `action,' `motion,' `movement,' `emotion,' and `expressive.' Finally, I included the name of specific technologies that are often used in interactive systems: `motion capture,' `accelerometer,' `wii,' `kinect' and `leap motion.' 

The first round of analysis involved an OSX shell script crawling through the content of the PDF files using the \texttt{mdfind -count} command. This terminal command returns a spotlight search based on the OSX index of the files. Some random control checks were done to validate the quality of the returned result. Finally, a spreadsheet was used to calculate the percentages and lay out the values in Table~\ref{Jensenius:tab:NIME}. 


\subsection{Results}

There are several interesting findings from Table~\ref{Jensenius:tab:NIME}: 

\begin{itemize}

\item There are some, but very few (1~\%), NIME papers that do not contain the word `music'

\item `Gesture' is used on average in 62~\% of all NIME papers, with only minor fluctuations from year to year

\item The motion-related terms (`action,' `motion,' `movement,') are used in about 50\% of the papers, also with only minor fluctuations over the years

\item `Expressive' is used in 49\% of the papers, while `emotion' is used in only 18\%

\item `Motion capture' and `accelerometer' are used evenly throughout the years, while `wii,' `kinect' and `leap motion' show up only as they were introduced to the market (2007, 2011, 2013, respectively)

\end{itemize}

It is particularly interesting to see that `gesture' is, in fact, the most commonly used of the terms, after `music.'


\section{`Gesture' Elsewhere}
\label{Jensenius:complimentaryresults}

To compare the terms mentioned above to other related conferences and journals, I carried out a second study.


\subsection{Method}

The proceedings of the \emph{Sound and Music Computing} (SMC) conference are freely available online as collections of PDF files,\footnote{\url{http://smcnetwork.org/resources/smc_papers/}} 
and it was therefore easy to download and analyse this collection in the same way as I did the NIME corpus. 

The proceedings of the \emph{International Computer Music Conference} (ICMC) are not freely available, but it is possible to search the full bibliography of all ICMC papers online.\footnote{\url{http://quod.lib.umich.edu/i/icmc/}} In this case, then, I had to perform manual searches for each of the terms. This produced only information about the total number of papers containing the terms, and I was not able to break down the numbers to annual figures. 

To complement the results with some data from the HCI community, I also did manual searches within the library containing \emph{Publications from ACM and Affiliated Organizations}\footnote{\url{http://dl.acm.org/results.cfm?&query=&dlr=ACM}}
and the large collection of the \emph{ACM Guide to Computing Literature}.\footnote{\url{http://dl.acm.org/results.cfm?&query=&dlr=GUIDE}} Finally, the \emph{Archive of the Journal of Biomechanics}\footnote{\url{http://www.sciencedirect.com/science/journal/00219290}} was also included to give a sense of how the term is used in biomechanics and kinesiology. 


\subsection{Results}

From the results, summarised in Table~\ref{Jensenius:tab:other}, we can see that `gesture' is used much more at NIME than at SMC  (62~\% vs 34~\%). This was to be expected, as SMC is less focused on instruments and performance than NIME. However, several of the motion-related terms are used almost as much at SMC as at NIME, so clearly there is a linguistic difference in play here. The underlying data also shows that there is no significant change in the use of the terms over time, which resonates with the profile of NIME. 

Even though the percentage values of the use of `gesture' at ICMC are much lower than at NIME (17~\% vs 62~\%), the actual number of papers using the term is almost the same. This could be attributed to the fact that ICMC has overlapped considerably with the NIME community over the last decade. Strangely, though, the technology terms generated very low values at ICMC (less than 3~\%). This could be an indication that `gesture' is being used more in a metaphorical sense at ICMC, although the underlying data is too weak to draw a clear conclusion in this regard. 

Looking at the results from the HCI community, `action' is by far the most prominent of the terms in the ACM libraries (11~\% and 22~\%). `Action' is also widely used in the biomechanics community (38~\%), but here `motion' and `movement' are used even more frequently (51~\% and 43~\%). All of the other terms generated fairly low percentage values, including, somewhat surprisingly, the technology terms. 


\section{Concordance and Collocation}

Along with simply counting papers mentioning a given term, it is useful to look at a concordance and collocation analysis of how the terms are being used.

\subsection{Method}

I extracted text of all of the PDF files in the NIME corpus into separate text files using CasualText. Next, the text files were cleaned up through a batch process in TexMate, removing all header information, weird characters and hyphens in the text. The text files were then imported into CasualConc, in which the analysis was carried out. 


\subsection{Results}

The concordance analysis shows that `gesture' is used a total of 4,211 times in the NIME corpus. The result of the collocation analysis is presented in Table~\ref{Jensenius:tab:collocation}; it shows that the five most commonly used words preceding `gesture' (L1) are `expressive,' `musical,' `hand,' `instrumental' and `physical.' This supports the claim that `gesture' is, in fact, used to describe both motion-like and metaphorical qualities. 

The five most commonly used words following directly after `gesture' (R1) are: `recognition,' `data,' `analysis,' `control,' and `sound.' It is particularly interesting to see that `sound' is by far the most commonly used second term (R2), as in the combination `gesture and sound.' 


\begin{table*}[t]
\centering
\ra{1.2}
\caption{Usage of terms in papers published in NIME Proceedings 2001--2013}
\vspace{3pt} \noindent
\begin{small}
\begin{tabular}{lrrrrrrrrrrrrr}
\midrule
\textbf{Year} & \multicolumn{1}{c}{\textbf{\#}} & \multicolumn{1}{l}{\textbf{music}} & \multicolumn{1}{l}{\textbf{gest-}} & \multicolumn{1}{l}{\textbf{acti-}} & \multicolumn{1}{l}{\textbf{moti-}} & \multicolumn{1}{l}{\textbf{move-}} & \multicolumn{1}{l}{\textbf{emo-}} & \multicolumn{1}{l}{\textbf{expre-}} & \multicolumn{1}{l}{\textbf{motion}} & \multicolumn{1}{l}{\textbf{accelero-}} & \multicolumn{1}{l}{\textbf{wii}} & \multicolumn{1}{l}{\textbf{kine-}} & \multicolumn{1}{l}{\textbf{leap}} \\
\textbf{} & \multicolumn{1}{l}{\textbf{}} & \multicolumn{1}{l}{\textbf{}} & \multicolumn{1}{l}{\textbf{ure}} & \multicolumn{1}{l}{\textbf{on}} & \multicolumn{1}{l}{\textbf{on}} & \multicolumn{1}{l}{\textbf{ment}} & \multicolumn{1}{l}{\textbf{tion}} & \multicolumn{1}{l}{\textbf{ssive}} & \multicolumn{1}{l}{\textbf{capture}} & \multicolumn{1}{l}{\textbf{meter}} & \multicolumn{1}{l}{\textbf{}} & \multicolumn{1}{l}{\textbf{ct}} & \multicolumn{1}{l}{\textbf{motion}} \\
\toprule  
2001 & 14 & 100\% & 64\% & 57\% & 50\% & 64\% & 14\% & 57\% & 29\% & 29\% & 0\% & 0\% & 0\% \\ 
2002 & 48 & 100\% & 65\% & 52\% & 58\% & 65\% & 17\% & 60\% & 23\% & 15\% & 0\% & 0\% & 0\% \\ 
2003 & 48 & 100\% & 71\% & 35\% & 40\% & 48\% & 19\% & 50\% & 15\% & 17\% & 0\% & 0\% & 0\% \\ 
2004 & 54 & 100\% & 56\% & 37\% & 39\% & 54\% & 22\% & 43\% & 22\% & 20\% & 0\% & 0\% & 0\% \\ 
2005 & 75 & 100\% & 63\% & 48\% & 45\% & 56\% & 23\% & 48\% & 23\% & 24\% & 0\% & 0\% & 0\% \\ 
2006 & 81 & 100\% & 64\% & 41\% & 36\% & 52\% & 7\% & 41\% & 23\% & 16\% & 0\% & 0\% & 0\% \\ 
2007 & 103 & 100\% & 55\% & 38\% & 40\% & 57\% & 20\% & 50\% & 18\% & 17\% & 4\% & 0\% & 0\% \\ 
2008 & 87 & 100\% & 60\% & 52\% & 52\% & 59\% & 14\% & 45\% & 25\% & 22\% & 16\% & 0\% & 0\% \\ 
2009 & 110 & 90\% & 54\% & 36\% & 37\% & 51\% & 12\% & 35\% & 13\% & 26\% & 12\% & 0\% & 0\% \\ 
2010 & 111 & 100\% & 66\% & 50\% & 44\% & 55\% & 23\% & 45\% & 26\% & 26\% & 14\% & 0\% & 0\% \\ 
2011 & 130 & 100\% & 67\% & 60\% & 45\% & 59\% & 16\% & 51\% & 25\% & 26\% & 13\% & 5\% & 0\% \\ 
2012 & 129 & 99\% & 63\% & 61\% & 44\% & 57\% & 16\% & 53\% & 26\% & 33\% & 10\% & 12\% & 0\% \\ 
2013 & 118 & 99\% & 65\% & 51\% & 49\% & 64\% & 25\% & 56\% & 30\% & 26\% & 11\% & 22\% & 1\% \\ \hline
\textbf{Mean} & \textbf{85} & \textbf{99\%} & \textbf{62\%} & \textbf{48\%} & \textbf{45\%} & \textbf{57\%} & \textbf{18\%} & \textbf{49\%} & \textbf{23\%} & \textbf{23\%} & \textbf{6\%} & \textbf{3\%} & \textbf{0\%} \\
\textbf{Stdev} & \textbf{36} & \textbf{3\%} & \textbf{5\%} & \textbf{9\%} & \textbf{6\%} & \textbf{5\%} & \textbf{5\%} & \textbf{7\%} & \textbf{5\%} & \textbf{6\%} & \textbf{6\%} & \textbf{7\%} & \textbf{0\%} \\ 
\midrule
\end{tabular}
\end{small}
\label{Jensenius:tab:NIME}
\end{table*}



\begin{table*}[t]
\centering
\ra{1.2}
\caption{Usage of terms in papers in different conference series}
\vspace{3pt} \noindent
\begin{small}
\begin{tabular}{lrrrrrrrrrr}
\midrule
\multicolumn{1}{l}{\textbf{Conference}} & \multicolumn{1}{c}{\textbf{\#}} & \multicolumn{1}{l}{\textbf{music}} & \multicolumn{1}{l}{\textbf{gest-}} & \multicolumn{1}{l}{\textbf{acti-}} & \multicolumn{1}{l}{\textbf{moti-}} & \multicolumn{1}{l}{\textbf{move-}} & \multicolumn{1}{l}{\textbf{emo-}} & \multicolumn{1}{l}{\textbf{expre-}} & \multicolumn{1}{l}{\textbf{motion}} & \multicolumn{1}{l}{\textbf{accelero-}} \\ \multicolumn{1}{l}{\textbf{}} & \multicolumn{1}{c}{\textbf{}} 		& \multicolumn{1}{l}{\textbf{}} & \multicolumn{1}{l}{\textbf{ure}} & \multicolumn{1}{l}{\textbf{on}} & \multicolumn{1}{l}{\textbf{on}} & \multicolumn{1}{l}{\textbf{ment}} & \multicolumn{1}{l}{\textbf{tion}} & \multicolumn{1}{l}{\textbf{ssive}} & \multicolumn{1}{l}{\textbf{capture}} & \multicolumn{1}{l}{\textbf{meter}} \\
\toprule  
NIME 			& 1 108 	& 99\% 	& 62\% 	& 48\% 	& 44\% 	& 57\% 	& 18\% 	& 48\% 	& 23\%	 & 24\% \\ 
SMC 			& 601 		& 100\% 	& 34\% 	& 42\% 	& 31\% 	& 46\% 	& 19\% 	& 33\% 	& 15\%	 & 8\% \\ 
ICMC 			& 3 687 	& 100\% 	& 17\% 	& 17\% 	& 20\% 	& 24\% 	& 4\% 		& 20\% 	& 2\%	 & 3\% \\ 
ACM + Aff. 		& 399 664 	& 4\% 		& 3\% 		& 22\% 	& 8\% 		& 10\% 	& 2\% 		& 4\% 		& 3\%	 & 1\% \\ 
ACM Guide 		& 2 193 894 & 2\% 		& 1\% 		& 11\% 	& 5\% 		& 5\% 		& 1\% 		& 2\% 		& 1\%	 & 0.4\% \\ 
J. biomechanics & 18 193 	& 0.3\% 	& 0.2\% 	& 38\% 	& 51\% 	& 43\% 	& 0.1\% 	& 0.03\%	& 6\%	 & 4\%  \\ 
\midrule
\end{tabular}
\end{small}
\label{Jensenius:tab:other}
\end{table*}



\begin{table*}[htbp]
\centering
\ra{1.2}
\caption{Selected terms collocated with the 4211 instances of `gesture' in all NIME papers (2001--2013)}
\vspace{3pt} \noindent
\begin{small}
\begin{tabular}{lcrrrrrrcllllll}
\midrule
\multicolumn{1}{l}{\textbf{Word}} & \multicolumn{1}{c}{\textbf{LR total}} & \multicolumn{1}{c}{\textbf{L total}} & \multicolumn{1}{c}{\textbf{L5}} & \multicolumn{1}{c}{\textbf{L4}} & \multicolumn{1}{c}{\textbf{L3}} & \multicolumn{1}{c}{\textbf{L2}} & \multicolumn{1}{c}{\textbf{L1}} & \multicolumn{1}{c}{\textbf{Gesture}} & \textbf{R1} & \textbf{R2} & \textbf{R3} & \textbf{R4} & \textbf{R5} & \textbf{R total} \\
\toprule  
sound & 402 & 87 & 15 & 22 & 21 & 18 & 11 & 0 & 55 & 136 & 59 & 46 & 19 & 315 \\
recognition & 378 & 34 & 4 & 9 & 12 & 6 & 3 & 0 & 284 & 3 & 47 & 7 & 3 & 344 \\
control & 243 & 65 & 13 & 13 & 11 & 4 & 24 & 0 & 84 & 27 & 23 & 17 & 27 & 178 \\
musical & 214 & 110 & 12 & 20 & 5 & 5 & 68 & 0 & 12 & 38 & 26 & 12 & 16 & 104 \\
data & 212 & 47 & 13 & 12 & 13 & 7 & 2 & 0 & 119 & 9 & 17 & 8 & 12 & 165 \\
mapping & 210 & 86 & 6 & 8 & 18 & 36 & 18 & 0 & 55 & 17 & 33 & 13 & 6 & 124 \\
music & 189 & 47 & 5 & 11 & 7 & 18 & 6 & 0 & 30 & 24 & 62 & 15 & 11 & 142 \\
analysis & 173 & 45 & 6 & 7 & 23 & 8 & 1 & 0 & 94 & 5 & 11 & 13 & 5 & 128 \\
system & 149 & 52 & 10 & 10 & 20 & 12 & 0 & 0 & 1 & 62 & 11 & 10 & 13 & 97 \\
expressive & 137 & 115 & 14 & 4 & 6 & 5 & 86 & 0 & 1 & 5 & 9 & 4 & 3 & 22 \\
time & 129 & 85 & 8 & 15 & 20 & 9 & 33 & 0 & 3 & 4 & 10 & 18 & 9 & 44 \\
interface & 126 & 52 & 6 & 7 & 11 & 25 & 3 & 0 & 31 & 20 & 10 & 8 & 5 & 74 \\
interaction & 125 & 34 & 7 & 4 & 11 & 7 & 5 & 0 & 21 & 15 & 17 & 28 & 10 & 91 \\
performance & 119 & 48 & 7 & 5 & 10 & 9 & 17 & 0 & 3 & 17 & 12 & 27 & 12 & 71 \\
hand & 99 & 78 & 5 & 6 & 8 & 5 & 54 & 0 & 1 & 1 & 5 & 9 & 5 & 21 \\
audio & 96 & 36 & 4 & 9 & 9 & 14 & 0 & 0 & 3 & 30 & 9 & 10 & 8 & 60 \\
human & 96 & 40 & 4 & 6 & 2 & 6 & 22 & 0 & 0 & 25 & 4 & 16 & 11 & 56 \\
instrument & 96 & 52 & 12 & 11 & 13 & 6 & 10 & 0 & 4 & 8 & 14 & 11 & 7 & 44 \\
parameters & 96 & 24 & 5 & 9 & 6 & 4 & 0 & 0 & 33 & 6 & 12 & 7 & 14 & 72 \\
computer & 95 & 21 & 4 & 5 & 9 & 3 & 0 & 0 & 0 & 15 & 30 & 2 & 27 & 74 \\
physical & 95 & 68 & 2 & 9 & 2 & 4 & 51 & 0 & 2 & 11 & 4 & 4 & 6 & 27 \\
synthesis & 93 & 46 & 5 & 8 & 17 & 12 & 4 & 0 & 1 & 6 & 18 & 15 & 7 & 47 \\
processing & 90 & 18 & 3 & 4 & 6 & 3 & 2 & 0 & 26 & 22 & 9 & 6 & 9 & 72 \\
interactive & 87 & 30 & 11 & 4 & 4 & 2 & 9 & 0 & 1 & 10 & 11 & 16 & 19 & 57 \\
continuous & 73 & 47 & 6 & 4 & 5 & 11 & 21 & 0 & 2 & 6 & 4 & 8 & 6 & 26 \\
movement & 73 & 46 & 1 & 2 & 9 & 30 & 4 & 0 & 1 & 6 & 11 & 5 & 4 & 27 \\
motion & 71 & 46 & 4 & 5 & 3 & 25 & 9 & 0 & 3 & 7 & 3 & 7 & 5 & 25 \\
sensor & 71 & 46 & 6 & 9 & 22 & 9 & 0 & 0 & 8 & 6 & 1 & 4 & 6 & 25 \\
controlled & 65 & 13 & 3 & 6 & 3 & 1 & 0 & 0 & 44 & 4 & 1 & 2 & 1 & 52 \\
instrumental & 64 & 58 & 0 & 2 & 2 & 0 & 54 & 0 & 1 & 2 & 1 & 1 & 1 & 6 \\
performed & 64 & 40 & 5 & 2 & 5 & 1 & 27 & 0 & 5 & 9 & 6 & 1 & 3 & 24 \\
mappings & 62 & 21 & 2 & 2 & 4 & 13 & 0 & 0 & 8 & 15 & 12 & 3 & 3 & 41 \\
signal & 62 & 11 & 3 & 4 & 3 & 1 & 0 & 0 & 32 & 3 & 5 & 6 & 5 & 51 \\
action & 22 & 7 & 0 & 2 & 0 & 5 & 0 & 0 & 3 & 5 & 2 & 3 & 2 & 15 \\
accelerometer & 17 & 7 & 1 & 1 & 2 & 3 & 0 & 0 & 0 & 2 & 1 & 6 & 1 & 10 \\
emotion & 9 & 3 & 1 & 0 & 0 & 2 & 0 & 0 & 0 & 3 & 1 & 2 & 0 & 6 \\
wii & 5 & 0 & 0 & 0 & 0 & 0 & 0 & 0 & 0 & 0 & 2 & 2 & 1 & 5 \\ 
\midrule
\end{tabular}
\end{small}
\label{Jensenius:tab:collocation}
\end{table*}



\section{Conclusions}

These text-based analyses of papers published at NIME and related conferences and journals support the initial claim that `gesture' is a widely used term in the NIME community, more so than in related fields. The collocation analysis further documents that `gesture' is used together with a large number of other terms, including motion-like, technological, and metaphorical terms. 
These findings indicate that more care should be devoted to defining what is meant by `gesture' when it is used. It may also be worth using more precise alternatives when possible. For example, `hand motion' may be a better term than `gesture' when describing the physical motion of a pianist's hands. 
Such an effort could help prevent confusion within the NIME community and, not least, better explain what is meant by `gesture' when communicating with people from other fields of study. 
%That said, the multiple definitions and interpretations of `gesture' in the NIME community may also attest to the popularity and widespread usage in the NIME community. This is certainly positive, as it inspires a lot of interesting research within the community.  

Though limited in scope, this study has shown the possibilities of carrying out analyses on the NIME community through the proceedings corpus. In the future it would be interesting to carry out both larger collocation and concordance studies as well as more in-depth studies of how different terms are used in the community.


\section*{Author Commentary: Gestures at NIME---a revisited account}


\paragraph{Alexander Refsum Jensenius}


What is a \emph{gesture}? This has been an important research question for me for more than a decade. My main academic interest is to understand more about the musical importance of what I nowadays prefer to call \emph{music-related motion}. By this I mean any type of human body motion carried out in a musical context, whether it is the motion of \emph{performers}, such as sound-producing, sound-modifying and sound-accompanying actions, or the motion of \emph{perceivers}, anything from foot-tapping to dancing and air-instrument performance \cite{Gritten:2006,Gritten:2011,Jensenius:2010}. I want to understand more about how body motion contributes to our experience of music, and why physical motion can be perceived as \emph{gestures}. Thus a gesture refers to, in my definition, the \emph{meaning} attached to or perceived from a bodily action, not the action itself \cite{McNeill:2005}. Furthermore, I believe that our perception of `musical gestures' can be triggered both from physical motion or from sound, or a combination of the two.

Why am I so picky about the differences between motion, action and gesture? Mainly because of the problems I have experienced with using the term sloppily earlier in my research career. At some point I used the term `gesture' in every second sentence I wrote, most often referring to what I today would call `motion' or `action,' but often also mixing in some meaning-bearing components. I was quite confused about the term myself, and probably confused many others. At some point, most likely inspired by Marcelo Wanderley's focus on precise terminology \cite{Cadoz:2000}, I therefore started asking people what they meant with `gesture.' The responses varied considerably! After doing such ad-hoc enquiries for a couple of years, I finally decided to do a study of the usage of `gesture,' and this led to the NIME 2014 paper included in this volume.

Since there had already been written several literature reviews on the topic \cite{Cadoz:2000,Jensenius:2010}, I decided to employ a text mining approach using the NIME proceedings archive as the source material. The analysis showed that `gesture' was actually used in 62\% of all NIME papers to date, much more than other important terms (besides music), and much more than in any of the `control' communities. Furthermore, a collocation analysis of the words that `gesture' appeared with indicated that its usage varied considerably, from the description of concrete human motion in system control to auditory metaphors.

Obviously, a rough statistical analysis of 1108 papers does not leave room for a deeper interpretation of all its meanings, and I really hope that I (or someone else) can do a more thorough analysis and reflection in the future. That said, the empirical investigation did confirm that the term `gesture' is used a lot, is used by a lot of people, and is used with a lot of different meanings. Then one can ask if this is really a problem? After all, the NIME community strives for the development of music technologies that are more expressive and meaningful, hence we do want to develop interfaces that allow for the creation and perception of meaningful gestures. But on our way to understanding expressivity, we would typically use `motion capture'and `action recognition' methods, rather than `gesture capture' and `gesture recognition.'

One of the most intriguing features of the NIME community, including not only regular NIME conference participants but also everyone else that could have participated, is the inter/multi/cross/disciplinarity of the field. People come from a multitude of academic disciplines, but also from different parts of the creative industries and performing arts. As such, the terminology used vary considerably, and therefore I believe it is important to be conscious about the terms we use.

Throughout the years I have interacted extensively with researchers from fields such as biomechanics, kinesiology and physiotherapy, in which the term `gesture' is barely used at all. This is, they argue, because they are studying the human body as a (bio)mechanical system. Linguists and psychologists, on the other hand, do use the term `gesture' a lot, but then again their main research interest is that of human communication \cite{Gritten:2006}. I believe that both of these communities, the biomechanics and linguists that is, can be of great importance when it comes to moving the NIME community forwards in the quest for `gesture recognition systems.' This, however, requires that we are able to talk to them with terms that they understand.

Summing up, I hope my paper may create a heightened awareness for the term `gesture' in particular, but also a more general need for precise definitions of important terms. I also hope that more people can be inspired to carry out meta-studies of the NIME community itself through the proceedings archive. While I did not have space for it in my paper,
such studies could preferably combine text mining methods with more
qualitative studies.

\section*{Expert Commentary: `Gesture' in NIME: A Past, Present and Future Research topic}

\paragraph{Baptiste Caramiaux}

Body gestures have been at a focal point for a wide range of research in human (to human) communication and human–computer interaction (HCI) \cite{Kendon:2004,Kurtenbach:1990}. The choice behind the use of gesture in human communication and HCI is motivated by the need to differentiate between a goal-oriented physical movement, typically termed action, and a physical movement expressing a meaning, an instruction, an idea, or an emotion, typically termed gesture. In HCI, for instance, a gesture refers to shapes performed on a device and used to activate menus or trigger commands. 

In NIME, a research field at the intersection of HCI and Music, the term gesture has also been regularly used since its origin  \cite{Cadoz:2000}. No prior work has examined if gesture can be seen as the established term, or at least empirically established, to designate physical motion in NIME performance and research, and if the term has the same definition as in HCI or human communication. The article by Jensenius brings, according to me, two important elements to this research question, that I am proposing to comment on in the following. 

The first aspect is that the term gesture is usually preferred to other terminology such as action, motion or movement. This result is an important confirmation of our intuition as NIME researcher and practitioner. In my view, however, the article lacks an interpretation of this result. I believe that its use reflects a common motivation across research fields to consider physical movements that convey an expression of an idea, an emotion or a meaning. And consequently, I think that the use of gesture in the context of NIME is appropriate. The motivation behind researching and building new interfaces for musical expression is for someone to express musical ideas through them, in other words articulating a musical ‘discourse', for which gestures are an embodied medium. 

I shall note that the author proposes an explicit comparison with the HCI community by inspecting the occurrences of gesture within prior work in this specific field. However the approach is not able to offer further insights on cross-disciplinary analysis. The NIME community has a culture anchored in performance and embodiment (as illustrated by the high percentage of occurrence of the term gesture in the NIME proceedings) while HCI encompasses a much broader range of research that may not involve any gestures or body movements. 

The second aspect is that the use of the term gesture appears to be rather heterogeneous. The method chosen by the author to investigate this is based on an analysis of the recurrent terms collocated to the term gesture, which shows the variety in these collocated terms. The method results in an interesting lexical field of the use of gesture. Further research is then needed to provide more semantic insights. Averaged statistics on word collocation narrow the scope of interpretation and discussion of the results, making hard for a NIME researcher or practitioner to fully understand how the term gesture is actually used within the community and what it means. 

Having conducted myself research on the role of gesture in music performance and sonic interaction design \cite{Caramiaux:2012}, I am aware of the difficulty of the task. Gesture has become a versatile term in NIME, whose meaning is hard to grasp, referring sometimes to physical movement, sometimes to a metaphorical shape in music (as also reported by Jensenius in his article). 

However, if gesture has gained such a versatile character, would it not be more pertinent to inspect how the term has evolved thanks to the community, so thanks to theory and practice? In other words, instead of collecting occurrences of the term over years, it may be preferable to categorise these occurrences and inspecting linguistic differences. I thus believe that such investigation would benefit from a broader analysis based on interaction with practitioners and researchers from the field. Objective measures, such as statistics of occurrences, must be confronted to subjective measures collected from experience within the community. Such work, I am convinced, would provide a solid grounding for further scientific research in NIME and thus contribute to the field's maturity.

In summary, this article gives an excellent starting point for a constructive discussion and documented research on gesture within NIME. In line with the author, I believe that, since the term is so present in NIME, further research must be conducted in that direction, but I would argue to involve the community itself, borrowing methodology from social sciences and HCI. In turn, this could lead to a step forward in the field's maturity.


%\end{document}
